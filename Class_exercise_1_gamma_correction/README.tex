%Introduction
%导言区,一般用于宏包的调用,对文档进行一些全局的设置
\documentclass[11pt,a4paper,fleqn]{article}

% \usepackage{ctex}%chinese pack
\usepackage{ulem}%better underline
\usepackage{amsmath} %math
\usepackage{amsfonts}
\usepackage{amssymb}
\usepackage{amsthm}
\usepackage{graphicx}
\usepackage{float}

\title{CS-GY-6533 Interactive Computer Graphics In-Class Exercise 1}
\author{Haotian Yi N18800809}
\date{\today}

%Main Body
%正文区
\begin{document}
\maketitle
%hello \heiti 你好

\section{Gamma Correction}
Codes for implementing gamma correction are below:
\begin{figure}[H]
	\centering
	\includegraphics[scale = 0.3]{code.png}
	%\title{code}
\end{figure}

Original picture:
\begin{figure}[H]
	\centering
	\includegraphics[scale = 0.4]{origin.png}
	%\title{code}
\end{figure}
Gamma: 4
\begin{figure}[H]
	\centering
	\includegraphics[scale = 0.3]{gamma_corrected4.png}
	%\title{code}
\end{figure}
Gamma: 0.25 (Nonlinear inverse from last correction)
\begin{figure}[H]
	\centering
	\includegraphics[scale = 0.3]{gamma_corrected25.png}
	%\title{code}
\end{figure}
Gamma: 0.25 (correct with 0.25 for second time)
\begin{figure}[H]
	\centering
	\includegraphics[scale = 0.3]{gamma_corrected25x2.png}
	%\title{code}
\end{figure}


\end{document}

